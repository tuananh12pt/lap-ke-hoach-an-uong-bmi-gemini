% Overleaf-ready XeLaTeX template (Vietnamese)
% Save as main.tex and upload to Overleaf. Set compiler to XeLaTeX.
\documentclass[12pt]{article}

% Language and fonts (XeLaTeX)
\usepackage{fontspec}
\setmainfont{Times New Roman} % change if you prefer another font that supports Vietnamese
\usepackage{polyglossia}
\setdefaultlanguage{vietnamese}

% Useful packages
\usepackage{amsmath,amssymb,amsthm}
\usepackage{graphicx}
\usepackage{booktabs}
\usepackage{geometry}
\usepackage{hyperref}
\usepackage{microtype}
\geometry{margin=1in}

% Bibliography with biblatex (biber backend)
\usepackage[backend=biber,style=authoryear]{biblatex}
\addbibresource{refs.bib}

% Document info
\title{Mẫu tài liệu - Overleaf (XeLaTeX)}
\author{Tên Tác Giả}
\date{\today}

\begin{document}
\maketitle

\begin{abstract}
Đây là một template tối giản sẵn dùng trên Overleaf. Chọn compiler là XeLaTeX để hỗ trợ UTF-8 và font hệ thống (tiếng Việt).
\end{abstract}

\section{Giới thiệu}
Viết một đoạn văn mẫu. Bạn có thể thay đổi font trong phần preamble bằng `\setmainfont{}`.

\section{Công thức toán}
Ví dụ inline: $e^{i\pi} + 1 = 0$.\
Display:
\[
  \int_{0}^{1} x^2 \, dx = \frac{1}{3}
\]
Nhiều dòng với amsmath:
\begin{align}
  a + b &= c \\
  d + e &= f
\end{align}

\section{Hình ảnh}
Đặt ảnh vào thư mục `images/` và dùng:
\begin{figure}[ht]
  \centering
  \includegraphics[width=0.6\textwidth]{images/sample.png}
  \caption{Ảnh minh họa}
  \label{fig:sample}
\end{figure}

\section{Bảng}
\begin{table}[ht]
  \centering
  \begin{tabular}{lcr}
    \toprule
    Tên & Số & Ghi chú \\
    \midrule
    A & 10 & ví dụ \\
    B & 20 & ví dụ 2 \\
    \bottomrule
  \end{tabular}
  \caption{Bảng ví dụ}
\end{table}

\section{Tham khảo}
Sử dụng `refs.bib` cùng `biblatex` (biber). Ví dụ citation: \cite{knuth1984texbook}.

\printbibliography

\end{document}
